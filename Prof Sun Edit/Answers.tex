\documentclass[11pt]{article}

\usepackage[margin=2cm]{geometry}
\usepackage{amsmath}

\begin{document}
\begin{enumerate}
\item \textbf{Page 15: testing set}

All those data are in my computer at the campus. They will be uploaded when I am back to school. Right now, only the codes and the results are available in my github repository.

\item \textbf{Page 25: Should we compare this with the non-core region}

 Done
 
\item \textbf{Page 32: I do not think this sentence is correct. AUC has different meaning.}

The interpretation of the AUC values were deleted. But, I think, to some extent, we can interpret the AUC as the probability to distinguish a randomly chosen positive sample from a randomly chosen negative sample. The arguments are as follows.

Suppose $M$ is the model, $x$ is the input, $T$ is the threshold. If $M(x)>T$, $x$ is classified as positive. Otherwise, $x$ is classified as negative. $FPR$ and $TPR$ can be viewed as two functions of $T$, given by 
$$TPR(T)=\int_{T}^{\infty}f_0(y)dy$$
$$FPR(T)=\int_{T}^{\infty}f_1(y)dy$$
Here, $f_0$ is the distribution density of the set $\{M(x): x\; \text{ is a positive sample}\}$, and $f_1$ is the distribution density of the set $\{M(x): x\; \text{ is a negative sample}\}$.
The $AUC$ is given by the following equations
\begin{align*}
AUC &= -\int_{-\infty}^{\infty}TPR(T)\; d(FPR(T))\\
	&= \int_{-\infty}^{+\infty}TPR(T)f_1(T)dT\\
	&= \int_{-\infty}^{+\infty}\int_{T}^{\infty}f_0(y)f_1(T)dydT\\
	&= \int_{-\infty}^{+\infty}\int_{-\infty}^{\infty}I(y>T)f_0(y)f_1(T)dydT\\
	&= \int_{-\infty}^{+\infty}\int_{-\infty}^{\infty}I(y_0>y_1)f_0(y_0)f_1(y_1)dy_0dy_1\\
	&= P(y_0>y_1)\\
	&= P(M(x_0)>M(x_1))\\
\end{align*}
Here, $y_0=M(x_0)$, $y_1=M(x_1)$, where $x_0$ is a randomly chosen positive sample and $x_1$ is a randomly chosen negative sample. 

\item \textbf{Page 33: Are there methods from other investigators to achieve this task? If yes, you should compare with them. I do not know the field well enough to give references and which methods to compare with. But my gut feeling is that there should be some you can compare with.}
 

As far as I know, there is no research specifically focused on this task. There are researches slightly related to this task, as introduced in the introduction chapter, but they are aiming to predict epitopes. To some extent, this task is new. The method we used is new as well.

\item \textbf{Page 49: I cannot see data for this? Did you provide data to support this claim?}

Table 4.3 on page 45 can give some support for this claim.




\end{enumerate}



\end{document}